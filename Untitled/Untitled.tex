% !TeX program = pdfLaTeX
\documentclass[smallextended]{svjour3}       % onecolumn (second format)
%\documentclass[twocolumn]{svjour3}          % twocolumn
%
\smartqed  % flush right qed marks, e.g. at end of proof
%
\usepackage{amsmath}
\usepackage{graphicx}
\usepackage[utf8]{inputenc}

\usepackage[hyphens]{url} % not crucial - just used below for the URL
\usepackage{hyperref}
\providecommand{\tightlist}{%
  \setlength{\itemsep}{0pt}\setlength{\parskip}{0pt}}

%
% \usepackage{mathptmx}      % use Times fonts if available on your TeX system
%
% insert here the call for the packages your document requires
%\usepackage{latexsym}
% etc.
%
% please place your own definitions here and don't use \def but
% \newcommand{}{}
%
% Insert the name of "your journal" with
% \journalname{myjournal}
%

%% load any required packages here
\usepackage{amsmath}




\begin{document}

\title{Title here \thanks{Grants or other notes about the article that should go on the front page
should be placed here. General acknowledgments should be placed at the
end of the article.} }
 \subtitle{Do you have a subtitle? If so, write it here} 

    \titlerunning{Short form of title (if too long for head)}

\author{  Äüthör 1 \and  Âuthóř 2 \and  }

    \authorrunning{ Short form of author list if too long for running head }

\institute{
        Äüthör 1 \at
     Department of YYY, University of XXX \\
     \email{\href{mailto:abc@def}{\nolinkurl{abc@def}}}  %  \\
%             \emph{Present address:} of F. Author  %  if needed
    \and
        Âuthóř 2 \at
     Department of ZZZ, University of WWW \\
     \email{\href{mailto:djf@wef}{\nolinkurl{djf@wef}}}  %  \\
%             \emph{Present address:} of F. Author  %  if needed
    \and
    }

\date{Received: date / Accepted: date}
% The correct dates will be entered by the editor


\maketitle

\begin{abstract}
The text of your abstract. 150 -- 250 words.
\\
\keywords{
        key \and
        dictionary \and
        word \and
    }

    \subclass{
                    MSC code 1 \and
                    MSC code 2 \and
            }

\end{abstract}


\def\spacingset#1{\renewcommand{\baselinestretch}%
{#1}\small\normalsize} \spacingset{1}


\hypertarget{experiments}{%
\section{Experiments}\label{experiments}}

\hypertarget{comparison-of-the-indices}{%
\subsection{Comparison of the indices}\label{comparison-of-the-indices}}

Lets consider a set of indexes \(\mathscr{I} = {i_1, ..., i_n}\), a set
of datasaets \({\mathscr{D} = {d_1, ..., d_n}\) and a set of methods
\({\mathscr{M} = {m_1, ..., m_n}\) and a set of
\(\mathscr{K} = {k_1, ..., k_n}\) possible number of clusters.

For each method \(m\), taking into account all the possible pairs
\((i_1, i_2)\) we count the number of datasets \(d\) where the value of
optimal clusters for the dataset \(d\) according to the indice \(i_1\)
using the method \(m\) is equal to the value of optimal clusters or the
dataset \(d\) according to the indice \(i_2\) using the method \(m\).

\hypertarget{intro}{%
\section{Introduction}\label{intro}}

Your text comes here. Separate text sections with \cite{Mislevy06Cog}.

\hypertarget{sec:1}{%
\section{Section title}\label{sec:1}}

Text with citations by \cite{Galyardt14mmm}.

\hypertarget{sec:2}{%
\subsection{Subsection title}\label{sec:2}}

as required. Don't forget to give each section and subsection a unique
label (see Sect. \ref{sec:1}).

\hypertarget{paragraph-headings}{%
\paragraph{Paragraph headings}\label{paragraph-headings}}

Use paragraph headings as needed.

\begin{align}
a^2+b^2=c^2
\end{align}

\bibliographystyle{spphys}
\bibliography{bibliography.bib}

\end{document}
